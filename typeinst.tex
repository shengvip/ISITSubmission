\documentclass[runningheads,a4paper]{llncs}
\usepackage{amssymb}
\setcounter{tocdepth}{3}
\usepackage{graphicx}
\usepackage{url}
\newcommand{\keywords}[1]{\par\addvspace\baselineskip
\noindent\keywordname\enspace\ignorespaces#1}

\begin{document}

\mainmatter  % start of an individual contribution

% first the title is needed
\title{ three diminsional sound reproduction in Vehicles based on data mining techniques }
 % 基于数据挖掘技术的车内3D声场重建技术
% a short form should be given in case it is too long for the running head
% \titlerunning{Lecture Notes in Computer Science: Authors' Instructions}

% the name(s) of the author(s) follow(s) next
%
% NB: Chinese authors should write their first names(s) in front of
% their surnames. This ensures that the names appear correctly in
% the running heads and the author index.
%
\author{Maosheng Zhang, Ruimin Hu, Lin JIang}
% \thanks{Project imformation}
%
% \authorrunning{Lecture Notes in Computer Science: Authors' Instructions}
% (feature abused for this document to repeat the title also on left hand pages)
% the affiliations are given next; don't give your e-mail address
% unless you accept that it will be published
\institute{computer school, wuhan university\\
Bayi road, Wuhan, China\\
}
%

\maketitle

\begin{abstract}
The abstract should summarize the contents of the paper and should
contain at least 70 and at most 150 words. It should be written using the
\emph{abstract} environment.
\keywords{Vehicles, sound, data mining, reproduction}
\end{abstract}


\section{Introduction}\label{sec:Intro}
%汽车音频研究
Sound systems for vehicle have been well researched by scientists and engineers.   Akitoshi Yamada developed a sound reproduction system for vehicle using only a pair of loudspeakers in 1982\cite{Akito82}. Honda designed a sound reproducing apparatus for vehicle\cite{terai1990sound}. 

A sound reproducing apparatus for use in a vehicle having a speaker system constituted by at least one acoustic duct and a speaker unit disposed at a throat of the acoustic duct, an opening formed at the other end of the acoustic duct facing the passenger compartment of a vehicle. The apparatus is arranged such that, F0 and Fs are substantially equal to each other. Where Fs is the lowest frequency in a frequency range within which the imaginary part of the acoustic impedance at the sound radiating surface of the speaker system in the direction of the space within the passsenger compartment is zero and F0 is the low-range resonance frequency of the speaker system itself in the free space.

The system comprised a transfer function, a delay circuit, and a reverberation circuit. With the help of these components, a surrounding sound system was implemented. In 2003, Takeshi reproduced a requiered sound image for the specified seat with a sound system consisting of two loudspeakers for Vehicles\cite{Takeshi03}. In addition, sound systems using more than two loudspeakers are developed to generate surrounding ambiance acoustic effects\cite{clark1998vehicle}\cite{orellana2015loudspeaker}. FORD Motor Company invented a multichannel sound reproduction system for vehicles and applied for a patent in 2017. \cite{orellana2015loudspeaker}. The embodiments mentioned in this papent compposed of several loudspeakers, including a low-frequency loudspeaker or sub-woofer, placing in pilars, door frames and vehicle roof. 
%3D汽车音频

A vehicle audio system includes overhead speakers connected to an audio source through a control circuit. The control circuit effects signal delays, crossover filtering, and equalization. The overhead speakers are connected to the control circuit and are positioned such that they provide strong front staging and desired surrounding ambiance responsive to the output signals of the control circuit. According to one aspect of the invention, the speakers are mounted directly to the headliner.
%3D音频
The most there popular 3-D sound reproduction algorithms include Wave Field Synthesis(WFS), Ambisonics and Amplitude panning. WFS aims to reproduce the whole sound field and thus the real sound immersion was reproducted. However, WFS method is not practicle since there are too many loudspeakers requiered in WFS system. Ambisonics system  


\section{tree regression}\label{sec:regression}

\section{three diminsional sound reproduction algorithm in Vehicles}\label{sec:algorithm}

\section{experiment}\label{sec:experiment}

\section{conclusion}\label{sec:conclusion}

\label{bib:bibliography}

\bibliographystyle{ISIT}
\bibliography{ISIT}


\end{document}
