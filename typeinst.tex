\documentclass[runningheads,a4paper]{llncs}
\usepackage{amssymb}
\setcounter{tocdepth}{3}
\usepackage{graphicx}
\usepackage{url}
\newcommand{\keywords}[1]{\par\addvspace\baselineskip
\noindent\keywordname\enspace\ignorespaces#1}

\begin{document}

\mainmatter  % start of an individual contribution

% first the title is needed
\title{ three diminsional sound reproduction in Vehicles based on data mining techniques }
 % 基于数据挖掘技术的车内3D声场重建技术
% a short form should be given in case it is too long for the running head
% \titlerunning{Lecture Notes in Computer Science: Authors' Instructions}

% the name(s) of the author(s) follow(s) next
%
% NB: Chinese authors should write their first names(s) in front of
% their surnames. This ensures that the names appear correctly in
% the running heads and the author index.
%
\author{Maosheng Zhang, Ruimin Hu, Lin JIang}
% \thanks{Project imformation}
%
% \authorrunning{Lecture Notes in Computer Science: Authors' Instructions}
% (feature abused for this document to repeat the title also on left hand pages)
% the affiliations are given next; don't give your e-mail address
% unless you accept that it will be published
\institute{computer school, wuhan university\\
Bayi road, Wuhan, China\\
}
%

\maketitle

\begin{abstract}
The abstract should summarize the contents of the paper and should
contain at least 70 and at most 150 words. It should be written using the
\emph{abstract} environment.
\keywords{Vehicles, sound, data mining, reproduction}
\end{abstract}


\section{Introduction}\label{sec:Intro}
%汽车音频研究
Sound systems for vehicle have been well researched by scientists and engineers.   Akitoshi Yamada developed a sound reproduction system for vehicle using only a pair of loudspeakers in 1982\cite{Akito82}. The system comprised a transfer function, a delay circuit, and a reverberation circuit. With the help of these components, a surrounding sound system was implemented. 
Honda Motor designed a sound reproducing apparatus for vehicle in 1990\cite{terai1990sound}. The apparatus takes advantage of a acoustic duct and a loudspeaker placing in the duct. 
In 2003, Takeshi reproduced a requiered sound image for the specified seat with a sound system consisting of two loudspeakers for Vehicles\cite{Takeshi03}. 
In addition, sound systems using more than two loudspeakers are developed to generate surrounding ambiance acoustic effects\cite{clark1998vehicle}\cite{orellana2015loudspeaker}. A sound entertainment system for determined positions in a vehicle is proposed by David in 2007. This system provides ultrasonic waves and cancels the unwanted noise\cite{David07}. 
FORD Motor Company invented a multichannel sound reproduction system for vehicles and applied for a patent in 2017\cite{orellana2015loudspeaker}. The embodiments mentioned in this papent compposed of several loudspeakers, including a low-frequency loudspeaker or sub-woofer, placing in pilars, door frames and vehicle roof. Obviously, the sound reproduction system for entertainment in vehicle are well researched. Lots of patents about sound reproduction in vehicle are applied by vehicle companies to recreat acoustic environment\cite{Simon2005}\cite{Miriam2014}\cite{David07}\cite{Gibson15}. However, sound spatial perception is far from satisfactory and acoustic virtual reality has not been implemented in vehicle. Three-dimensional (3D) sound reproduction system provides immersive perception about sound sources and thus enhances the sensation of reality\cite{AasthaTASLP11}\cite{Danilo15TMM}\cite{zms2015}. It is necessary to reproduce 3D sound to enjoy realistic acoustic environment and sound events in vehicle.

%3D音频
The most there popular 3-D sound reproduction algorithms include Wave Field Synthesis(WFS), Ambisonics and Amplitude panning. WFS is based on Huyghens principle and it is able to reproduce the whole sound field and thus the real sound immersion was recreated\cite{Gergely17}. However, WFS method is not practicle since there are too many loudspeakers requiered in WFS system. Ambisonics system reproduces the sound pressure at listening point in the center of spherical loudspeaker array. While, it is impossible to configure a spherical loudspeaker array in vehicle. Amplitude panning is a widely used sound reproduction technique due to its computational efficiency. Vector base amplitude panning (VBAP) is a popular sound reproduction technique to render the sound direction and distance. And thus VBAP is considered as a promising technique to recreat sound events\cite{Pulkki01spatial}. Unfortunately, VBAP system requiers a spherical loudspeaker array, which is not satisfied in vehicle. 

Though there are several metods to reproduce sound field in physics or mathematics method, the intrinsical goal of sound field reproduction is to reproduce sound pressure at every point in the listening field. 


essentially


\section{tree regression}\label{sec:regression}

\section{three diminsional sound reproduction algorithm in Vehicles}\label{sec:algorithm}

\section{experiment}\label{sec:experiment}

\section{conclusion}\label{sec:conclusion}

\label{bib:bibliography}

\bibliographystyle{ISIT}
\bibliography{ISIT}


\end{document}
